%allgemeine Formatangaben
\documentclass[
 a4paper, 										% Papierformat
 12pt,												% Schriftgröße
 ngerman, 										% für Umlaute, Silbentrennung etc.
 titlepage,										% es wird eine Titelseite verwendet
 oneside, 										% einseitiges Dokument
 captions=nooneline,					% einzeilige Gleitobjekttitel ohne Sonderbehandlung wie mehrzeilige Gleitobjekttitel behandeln
 numbers=noenddot,						% Überschriften-??Nummerierung ohne Punkt am Ende
 parskip=half,									% zwischen Absätzen wird eine halbe Zeile eingefügt
 ]{scrartcl}

% Anpassung an Landessprache
\usepackage[ngerman]{babel}	

\usepackage[T1]{fontenc}	
\usepackage[utf8]{inputenc}	
\usepackage{textcomp} 																% Euro-Zeichen und andere
\usepackage[babel,german=quotes]{csquotes}						% Anführungszeichen
\RequirePackage[ngerman=ngerman-x-latest]{hyphsubst} 	% erweiterte Silbentrennung

% Befehle aus AMSTeX für mathematische Symbole z.B. \boldsymbol \mathbb
\usepackage{amsmath,amsfonts}

% Zeilenabstände und Seitenränder 
\usepackage{setspace}
\usepackage{geometry}

% Einbinden von JPG-Grafiken
\usepackage{graphicx}

% zum Umfließen von Bildern
% Verwendung unter http://de.wikibooks.org/wiki/LaTeX-Kompendium:_Baukastensystem#textumflossene_Bilder
\usepackage{floatflt}

% Verwendung von vordefinierten Farbnamen zur Colorierung
% Palette und Verwendung unter http://kitt.cl.uzh.ch/kitt/CLinZ.CH/src/Kurse/archiv/LaTeX-Kurs-Farben.pdf
\usepackage[usenames,dvipsnames]{color} 

% Tabellen
\usepackage{array}
\usepackage{longtable}

% einfache Grafiken im Code
% Einführung unter http://www.math.uni-rostock.de/~dittmer/bsp/pstricks-bsp.pdf
\usepackage{pstricks}

% Quellcodeansichten
\usepackage{verbatim}
\usepackage{moreverb} 											% für erweiterte Optionen der verbatim Umgebung
% Befehle und Beispiele unter http://www.ctex.org/documents/packages/verbatim/moreverb.pdf
\usepackage{listings} 											% für angepasste Quellcodeansichten siehe
% Kurzeinführung unter http://blog.robert-kummer.de/2006/04/latex-quellcode-listing.html

% verlinktes und Farblich angepasstes Inhaltsverzeichnis
\usepackage[pdftex,
colorlinks=true,
linkcolor=InterneLinkfarbe,
urlcolor=ExterneLinkfarbe]{hyperref}
\usepackage[all]{hypcap}

% URL verlinken, lange URLs umbrechen
\usepackage{url}

% sorgt dafür, dass Leerzeichen hinter parameterlosen Makros nicht als Makroendezeichen interpretiert werden
\usepackage{xspace}

% Beschriftungen für Abbildungen und Tabellen
\usepackage{caption}

% Entwicklerwarnmeldungen entfernen
\usepackage{scrhack}

\newcommand{\qq}[1]{\glqq{#1\grqq{}}} %Gänsefüßchen

\onehalfspacing 							% 1,5facher Zeilenabstand

\definecolor{InterneLinkfarbe}{rgb}{0.1,0.1,0.3} 	% Farbliche Absetzung von externen Links
\definecolor{ExterneLinkfarbe}{rgb}{0.1,0.1,0.7}	% Farbliche Absetzung von internen Links

% Einstellungen für Fußnoten:
\captionsetup{font=footnotesize,labelfont=sc,singlelinecheck=true,margin={5mm,5mm}}
						

\title{Nutzerhandbuch:  Höchstgeschwindigkeitsschilder}
\subtitle{Projekt 8 - Digitale Bildverarbeitung}

\author{Andrej Lisnitzki,, Michael Horn\vspace{4cm}}
\date{\today}
\begin{document}
\maketitle

\section{Allgemeines}
In diesem Projekt war es das Ziel mit Hilfe von ImageJ eine Erkennung von Höchstgeschwindigkeitsschildern zu implementieren.
Die Implementierung erfolgte als ausführbares ImageJ-Plugin.

Zur Erkennung von Schildern wird zunächst ein Farberkennung realisiert und im Anschluss mit einer Hough-Transformation nach Kreisen gesucht.
Gefundene Kreise werden im nächsten Schritt durch eine OCR-Anwendung analysiert.
Werden dabei gültige Geschwindigkeitsinformationen geliefert, so wird das Schild als Höchstgeschwindigkeitsschild erkannt und entsprechend markiert.

Damit die Erkennung der Geschwindigkeitsschilder funktioniert, wird GOCR\footnote{\url{http://jocr.sourceforge.net/}} als ausführbare OCR-Anwendung benötigt.
Diese Anwendung muss in dem entsprechendem Ordner des Plugins hinterlegt sein.
Weiterhin muss diese Binärdatei ausführbar sein (Executiable Flag gesetzt).

Ergebnisse werden im Ordner der Eingangsdatei gespeichert.

\section{Voraussetzungen}
ImageJ muss mindestens mit der Java Version 7 arbeiten und der Compiler muss auf 1.7 gestellt werden!
Ältere Versionen werden nicht unterstützt!

GOCR muss im entsprechenden Unterordner des Plugins ausführbar sein.

\pagebreak
\section{Parameter}
Folgende Parameter werden benötigt, damit die Erkennung funktioniert\footnote{\textbf{Achtung:} Derzeit erfolgt keine Überprüfung der Eingangswerte auf Korrektheit!}
\begin{itemize}
	\item Pfad zur Bilddatei
	\begin{itemize}
		\item filepath=plugins/Project/result/vlcsnap-2016-05-04-14h27m18s219.png
	\end{itemize}
	\end{itemize}
HoughCircle-Parameter:
\begin{itemize}
	\item Kleinster Radius des zu erkennenden Kreises
	\begin{itemize}
		 \item MinimumRadius=10
	\end{itemize}
	\item Größter Radius des zu erkennenden Kreises
	\begin{itemize}
		\item MaximumRadius=50
	\end{itemize}
	\item Schrittweite der Radiuserhöhung bei der Erkennung
	\begin{itemize}
		\item IncrementRadius=2
	\end{itemize}
	\item Anzahl der zu erkennenden Kreise
	\begin{itemize}
		\item NumberOfCircles=6
	\end{itemize}
	\item Anzahl der Treffer, damit der Kreis erkannt wird (0 <= 8), es werden Senkrechte, Vertikale und Diagonale Punkte verwendet um einen Kreis zu definieren
	\begin{itemize}
		\item Hits=6
	\end{itemize}
\end{itemize}
\end{document}