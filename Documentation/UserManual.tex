\input{Header}						

\title{Nutzerhandbuch:  Höchstgeschwindigkeitsschilder}
\subtitle{Projekt 8 - Digitale Bildverarbeitung}

\author{Andrej Lisnitzki,, Michael Horn\vspace{4cm}}
\date{\today}
\begin{document}
\maketitle

\section{Allgemeines}
In diesem Projekt war es das Ziel mit Hilfe von ImageJ eine Erkennung von Höchstgeschwindigkeitsschildern zu implementieren.
Die Implementierung erfolgte als ausführbares ImageJ-Plugin.

Zur Erkennung von Schildern wird zunächst ein Farberkennung realisiert und im Anschluss mit einer Hough-Transformation nach Kreisen gesucht.
Gefundene Kreise werden im nächsten Schritt durch eine OCR-Anwendung analysiert.
Werden dabei gültige Geschwindigkeitsinformationen geliefert, so wird das Schild als Höchstgeschwindigkeitsschild erkannt und entsprechend markiert.

Damit die Erkennung der Geschwindigkeitsschilder funktioniert, wird GOCR\footnote{\url{http://jocr.sourceforge.net/}} als ausführbare OCR-Anwendung benötigt.
Diese Anwendung muss in dem entsprechendem Ordner des Plugins hinterlegt sein.
Weiterhin muss diese Binärdatei ausführbar sein (Executiable Flag gesetzt).

Ergebnisse werden im Ordner der Eingangsdatei gespeichert.

\section{Voraussetzungen}
ImageJ muss mindestens mit der Java Version 7 arbeiten und der Compiler muss auf 1.7 gestellt werden!
Ältere Versionen werden nicht unterstützt!

GOCR muss im entsprechenden Unterordner des Plugins ausführbar sein.

\pagebreak
\section{Parameter}
Folgende Parameter werden benötigt, damit die Erkennung funktioniert\footnote{\textbf{Achtung:} Derzeit erfolgt keine Überprüfung der Eingangswerte auf Korrektheit!}
\begin{itemize}
	\item Pfad zur Bilddatei
	\begin{itemize}
		\item filepath=plugins/Project/result/vlcsnap-2016-05-04-14h27m18s219.png
	\end{itemize}
	\end{itemize}
HoughCircle-Parameter:
\begin{itemize}
	\item Kleinster Radius des zu erkennenden Kreises
	\begin{itemize}
		 \item MinimumRadius=10
	\end{itemize}
	\item Größter Radius des zu erkennenden Kreises
	\begin{itemize}
		\item MaximumRadius=50
	\end{itemize}
	\item Schrittweite der Radiuserhöhung bei der Erkennung
	\begin{itemize}
		\item IncrementRadius=2
	\end{itemize}
	\item Anzahl der zu erkennenden Kreise
	\begin{itemize}
		\item NumberOfCircles=6
	\end{itemize}
	\item Anzahl der Treffer, damit der Kreis erkannt wird (0 <= 8), es werden Senkrechte, Vertikale und Diagonale Punkte verwendet um einen Kreis zu definieren
	\begin{itemize}
		\item Hits=6
	\end{itemize}
\end{itemize}
\end{document}